% packages
\usepackage{amsmath}
\usepackage{amsthm}
\usepackage{amssymb}
\usepackage{mdframed}
\usepackage{subcaption}
\usepackage{cite}
\usepackage[dvipsnames]{xcolor}
\usepackage{tikz}
\usepackage{float}
\usepackage{tikz-cd}
\usepackage{dashbox}
\usepackage[normalem]{ulem}
\usepackage{bbm}
\usepackage{fancyhdr}
\newcommand\graphnode[1]{\begin{tabular}{@{}c@{}}#1\end{tabular}}
\usepackage{pgfplots}
\pgfplotsset{compat=newest}
\usepgfplotslibrary{fillbetween}
\usepackage{environ}
\makeatletter
\newsavebox{\measure@tikzpicture}
\NewEnviron{scaletikzpicturetowidth}[1]{%
  \def\tikz@width{#1}%
  \def\tikzscale{1}\begin{lrbox}{\measure@tikzpicture}%
  \BODY
  \end{lrbox}%
  \pgfmathparse{#1/\wd\measure@tikzpicture}%
  \edef\tikzscale{\pgfmathresult}%
  \BODY
}
\makeatother
\usetikzlibrary{hobby}
\usetikzlibrary{shapes.misc}
\usepackage[backref=page]{hyperref}
\hypersetup{
    breaklinks=true,
    colorlinks=true,
    urlcolor=urlcolor,
    linkcolor=linkcolor,
    citecolor=citecolor,
    }

\usepackage{booktabs}
\usepackage{enumitem}
\setlist[itemize]{noitemsep}
\setlist[enumerate]{noitemsep}


% commands
% symbols =======================
% mathbb
\newcommand{\NN}{\mathbb{N}}
\newcommand{\ZZ}{\mathbb{Z}}
\newcommand{\QQ}{\mathbb{Q}}
\newcommand{\XX}{\mathbb{X}}
\newcommand{\RR}{\mathbb{R}}
\renewcommand{\SS}{\mathbb{S}}
\newcommand{\CC}{\mathbb{C}}
\newcommand{\KK}{\mathbb{K}}
\newcommand{\HH}{\mathbb{H}}
\newcommand{\EE}{\mathbb{E}}
\newcommand{\PP}{\mathbb{P}}
\newcommand{\WW}{\mathbb{W}}
% mathcal
\newcommand{\Aa}{\mathcal{A}}
\newcommand{\Bb}{\mathcal{B}}
\newcommand{\Rr}{\mathcal{R}}
\newcommand{\Cc}{\mathcal{C}}
\newcommand{\Dd}{\mathcal{D}}
\newcommand{\Ee}{\mathcal{E}}
\newcommand{\Hh}{\mathcal{H}}
\newcommand{\Ff}{\mathcal{F}}
\newcommand{\Gg}{\mathcal{G}}
\newcommand{\Ii}{\mathcal{I}}
\newcommand{\Ll}{\mathcal{L}}
\newcommand{\Nn}{\mathcal{N}}
\newcommand{\Mm}{\mathcal{M}}
\newcommand{\Oo}{\mathcal{O}}
\newcommand{\Pp}{\mathcal{P}}
\newcommand{\Qq}{\mathcal{Q}}
\newcommand{\Ss}{\mathcal{S}}
\newcommand{\Tt}{\mathcal{T}}
\newcommand{\Vv}{\mathcal{V}}
\newcommand{\Ww}{\mathcal{W}}
\newcommand{\Xx}{\mathcal{X}}
\newcommand{\Yy}{\mathcal{Y}}
\newcommand{\Zz}{\mathcal{Z}}
% letters
\newcommand{\al}{\alpha}
\newcommand{\be}{\beta}
\newcommand{\ga}{\gamma}
\newcommand{\de}{\delta}
\newcommand{\eps}{\varepsilon}
\renewcommand{\th}{\theta}
\newcommand{\la}{\lambda}
\newcommand{\si}{\sigma}
\newcommand{\om}{\omega}
\newcommand{\Ga}{\Gamma}
\newcommand{\De}{\Delta}
\newcommand{\La}{\Lambda}
\newcommand{\Si}{\Sigma}
\newcommand{\Om}{\Omega}
\renewcommand{\phi}{\varphi}
\renewcommand{\epsilon}{\varepsilon}

% operators ==========
\renewcommand{\ker}{\operatorname{Ker}}
\DeclareMathOperator*{\rk}{rk}
\DeclareMathOperator*{\diag}{diag}
\DeclareMathOperator*{\card}{card}
\DeclareMathOperator*{\sgn}{sgn}
\DeclareMathOperator*{\argmax}{arg\,max}
\DeclareMathOperator*{\argmin}{arg\,min}
\newcommand{\scal}[2]{\left\langle #1,\, #2 \right\rangle}
\newcommand{\enscond}[2]{\left\lbrace #1,\quad #2 \right\rbrace}
\newcommand{\pd}[2]{ \frac{ \partial #1}{\partial #2} }
\newcommand{\umin}[1]{\underset{#1}{\min}\;}
\newcommand{\umax}[1]{\underset{#1}{\max}\;}
\newcommand{\uargmin}[1]{\underset{#1}{\argmin}\;}
\newcommand{\uargmax}[1]{\underset{#1}{\argmax}\;}
\newcommand{\norm}[1]{\left\|#1\right\|}
\newcommand{\abs}[1]{\left|#1\right|}
\newcommand{\pa}[1]{\left(#1\right)}
\newcommand{\bra}[1]{\left[#1\right]}
\newcommand{\cro}[1]{\left\{#1\right\}}
\newcommand{\set}[1]{\left\{#1\right\}}
\DeclareMathOperator{\KL}{KL}
\newcommand{\KLdiv}[2]{\KL\pa{#1 | #2}}
\newcommand{\KLproj}{\text{Proj}^{\tiny\KL}}
\newcommand{\choice}[1]{ \left\{ \begin{array}{l} #1 \end{array} \right. }
\def\ones{\mathbbm{1}}
\def\id{\operatorname{id}}
\def\anti{\operatorname{anti-id}}
\newcommand{\graph}{\mathrm{gph}}
% text
\newcommand{\qandq}{\quad\text{and}\quad}
\newcommand{\qwhereq}{\quad\text{where}\quad}
\newcommand{\qifq}{\quad\text{if}\quad}
\newcommand{\qarrq}{\quad\Longrightarrow\quad}
\newcommand{\as}{\text{as }}
\newcommand{\qas}{\quad\text{as }}
\newcommand{\where}{\text{where }}
\newcommand{\tmin}{\text{min}}
\newcommand{\tmax}{\text{max}}

% mathrm
\newcommand{\dd}{\,\mathrm d}

% Mark sections of captions for referring to divisions of figures
\newcommand{\figleft}{{(Left)}}
\newcommand{\figcenter}{{(Center)}}
\newcommand{\figright}{{(Right)}}
\newcommand{\figtop}{{(Top)}}
\newcommand{\figbottom}{{(Bottom)}}
\newcommand{\captiona}{{(a)}}
\newcommand{\captionb}{{(b)}}
\newcommand{\captionc}{{(c)}}
\newcommand{\captiond}{{(d)}}

\newcommand{\defeq}{\triangleq}

\usepackage[capitalize]{cleveref}
\crefname{section}{Sec.}{Sec.}
\Crefname{section}{Section}{Sections}
\Crefname{table}{Table}{Tables}
\crefname{table}{Tab.}{Tabs.}
\crefname{equation}{}{}
\Crefname{equation}{Eq.}{}

\definecolor{tabblue}{rgb}{0.12156862745098039, 0.4666666666666667, 0.7058823529411765}
\definecolor{tabblue}{rgb}{0.12156862745098039, 0.4666666666666667, 0.7058823529411765}
\definecolor{taborange}{rgb}{1.0, 0.4980392156862745, 0.054901960784313725}
\definecolor{tabgreen}{rgb}{0.17254901960784313, 0.6274509803921569, 0.17254901960784313}
\definecolor{tabred}{rgb}{0.8392156862745098, 0.15294117647058825, 0.1568627450980392}
\definecolor{tabpurple}{rgb}{0.5803921568627451, 0.403921568627451, 0.7411764705882353}
\definecolor{tabbrown}{rgb}{0.5490196078431373, 0.33725490196078434, 0.29411764705882354}
\definecolor{tabpink}{rgb}{0.8901960784313725, 0.4666666666666667, 0.7607843137254902}
\definecolor{tabgray}{rgb}{0.4980392156862745, 0.4980392156862745, 0.4980392156862745}
\definecolor{tabolive}{rgb}{0.7372549019607844, 0.7411764705882353, 0.13333333333333333}
\definecolor{tabcyan}{rgb}{0.09019607843137255, 0.7450980392156863, 0.8117647058823529}
\hypersetup{
  breaklinks=true,
  colorlinks=true,
  linkcolor=taborange!75!black,
  urlcolor=tabblue,
  citecolor=tabgreen,
  }

% project specific ============
\newcommand{\GW}{\operatorname{GW}}
\newcommand{\gw}{\operatorname{GW}}
\newcommand{\W}{\operatorname{W}}
\newcommand{\supp}{\operatorname{supp}}
\newcommand{\tr}{\operatorname{tr}}
\newcommand{\pimon}{\pi_\text{mon}^\oplus}
\newcommand{\piantimon}{\pi_\text{mon}^\ominus}
\newcommand{\Tmon}{T_\text{mon}^\oplus}
\newcommand{\Tantimon}{T_\text{mon}^\ominus}
\newcommand{\push}{_*}
\newcommand{\pushonly}{*}
\newcommand{\opt}{^\star}


\newlist{abbrv}{itemize}{1}
\setlist[abbrv,1]{label=,labelwidth=1.2in,align=parleft,itemsep=0.0\baselineskip,leftmargin=!}
\def\symbolentry#1#2{\item[#1] #2}

\newcommand{\capcenter}{\textbf{(Center)\ }}
\newcommand{\capright}{\textbf{(Right)\ }}
\newcommand{\capleft}{\textbf{(Left)\ }}
\newcommand{\Pac}{\Pp_{\text{ac}}}
\newcommand{\Pactwo}{\Pp_{2,\text{ac}}}

\pgfplotsset{colormap={whitered}{color(0cm)=(white); color(1cm)=(tabpurple!80)}}

% environments ====================
% theorems
\def\newtheoremlines{\newmdtheoremenv[
    backgroundcolor=white,linecolor=black!40,
    linewidth=2pt,topline=false,rightline=false,leftline=true,bottomline=false,
    innertopmargin=-8pt,innerbottommargin=0pt,leftmargin=-12pt,rightmargin=-10pt,
    skipabove=0,skipbelow=0,
    % skipabove=\baselineskip,skipbelow=\baselineskip,
    usetwoside=false]}
\theoremstyle{plain}
\newtheoremlines{defi}{Definition}[chapter]
\theoremstyle{definition}
\newtheoremlines{proposition}{Proposition}[chapter]
\newtheoremlines{theorem}[proposition]{Theorem}
\newtheoremlines{coro}[proposition]{Corollary}
\newtheoremlines{lemma}[proposition]{Lemma}
\newtheoremlines{conjecture}[proposition]{Conjecture}
\newtheorem{example}{Example}[chapter]
\newtheorem{remark}{Remark}[chapter]


% todo ================
\usepackage[colorinlistoftodos,prependcaption,textsize=tiny,textwidth=2.7cm,disable]{todonotes}
\newcommand{\formva}[1]{\todo[linecolor=tabred,backgroundcolor=tabred!25,bordercolor=tabred]{\textcolor{tabred}{\textbf{For MVA:}} #1}}
\newcommand{\todoo}[1]{\todo[linecolor=tabred,backgroundcolor=tabred!25,bordercolor=tabred]{\textcolor{tabred}{\textbf{For MVA:}} #1}}
\newcommand{\forpaper}[1]{\todo[linecolor=taborange,backgroundcolor=taborange!25,bordercolor=taborange]{\textcolor{taborange}{\textbf{For paper:}} #1}}
\newcommand{\optional}[1]{\todo[linecolor=tabpurple,backgroundcolor=tabpurple!25,bordercolor=tabpurple]{\textcolor{tabpurple}{\textbf{Optional:}} #1}}
\newcommand{\info}[1]{\todo[linecolor=tabblue,backgroundcolor=tabblue!25,bordercolor=tabblue]{\textcolor{tabblue}{\textbf{Remark:}} #1}}
\newcommand{\question}[1]{\todo[linecolor=tabgreen,backgroundcolor=tabgreen!25,bordercolor=tabgreen]{\textcolor{tabgreen}{\textbf{Question:}} #1}}
\newcommand{\fx}[1]{\todo[linecolor=tabblue,backgroundcolor=tabblue!25,bordercolor=tabblue]{\textcolor{tabblue}{\textbf{Comment (FX):}} #1}}
\newcommand{\tl}[1]{\todo[linecolor=tabblue,backgroundcolor=tabblue!25,bordercolor=tabblue]{\textcolor{tabblue}{\textbf{Comment (TL):}} #1}}
\newcommand{\td}[1]{\todo[linecolor=tabblue,backgroundcolor=tabblue!25,bordercolor=tabblue]{\textcolor{tabblue}{\textbf{Comment (TD):}} #1}}

\usepackage{import}
\newcommand{\incfig}[3]{%
\begin{figure}[!h]
    \centering
    \def\svgwidth{#3\linewidth}
    \import{./img/}{#1.pdf_tex}
    \caption{#2}
\end{figure}
}

\usepackage{titlesec}

\usepackage{minitoc}
\usepackage{algorithm}


\title{
    \vspace{-4cm}
    \includegraphics[width=.3\linewidth]{img/logo-ens.jpg}
    \hspace{1cm}
    \includegraphics[width=.25\linewidth]{img/logo_mines.png}
    \\
    \vspace{1cm}
    \large \textsc{ENS Paris Saclay (MVA) \& Mines Paris}\\
    \large \textsc{Rapport de stage de fin d'études}\\
    \vspace{1cm}
    \par\noindent\rule{.8\textwidth}{0.4pt}\\
    \vspace{1mm}
    \huge \textbf{Existence of Monge maps for the Gromov--Wasserstein distance}\\
    \vspace{-4mm}
    \par\noindent\rule{.8\textwidth}{0.4pt}\\
    \vspace{1cm}
    \Large Théo \textsc{Dumont}\\
    \vspace{2cm}
    \includegraphics[width=.15\linewidth]{img/logo-ligm.png}\\
    \large \textsc{Laboratoire d'Informatique Gaspard Monge (\textsc{ligm})}\\
    \vspace{1cm}
    \begin{flushleft}
        \large \textit{Supervisor 1:} François-Xavier \textsc{Vialard}, \textsc{ligm}\\
        \large \textit{Supervisor 2:} Théo \textsc{Lacombe}, \textsc{ligm}\\
        \large \textit{Referent at MVA:} Gabriel \textsc{Peyré}, \textsc{cnrs, dma, ens}\\
        \large \textit{Referent at Mines:} Olivier \textsc{Hermant}, {Mines Paris}
    \end{flushleft}
    \vfill
    \large April 15\textsuperscript{th} 2022 -- September 30\textsuperscript{th} 2022
    % \vfill
}
\author{}
\date{}

\fancyhf{}
\fancyhead[L]{\thepage}
\fancyhead[R]{\leftmark}
\newcommand{\Chapter}[1]{\chapter{#1}{\hypersetup{linkcolor=black}\minitoc\vspace{5mm}}}
\dominitoc


\usepackage{titlesec}
\titleformat{\chapter}[display]{\normalfont\huge\bfseries}{\chaptertitlename\ \thechapter}{20pt}{\Huge}
\titlespacing*{\chapter}{0pt}{-40pt}{40pt}
\usepackage{algpseudocode}
\usepackage{multirow}
\usepackage{mathdots}
\usepackage{sidecap}
\usepackage{mathtools}